In recent years, railway transportation has experienced considerable progress,
where commercial train speed exceeds 300 km/h in many countries.
To achieve this speed, a huge amount of energy (e.g. electricity or gas) is consumed, and,
therefore, it is of great interest to study new methods since energy costs are rising. As
electric train are widely deployed worldwide, we focus here on the interaction between the line and
the train. Especially when considering high speeds, it is
necessary to maintain a constant flow of electricity on the
tensile member for proper operation of railway units. The
pantograph/catenary system based on overhead contact line
(OCL or catenary) is the most widely used for feeding modern trains. 
The catenary is a structure formed by cables that
supply electricity to the pantograph, which is responsible for
obtaining the energy by direct contact through a surface rub,
where the development of a rigorous mechanical calculation
is essential in order to achieve optimal coupling between both
members.

In \cite{}, we introduced the mechanical equations for solving the catenary-interaction problem and two
tools were presented.



The paper is structured as follows 
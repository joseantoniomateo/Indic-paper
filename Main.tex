\documentclass[final,1p,times]{elsarticle}

\usepackage{times}
\usepackage[T1]{fontenc}
\usepackage[utf8]{inputenc}
\usepackage{epsfig}
\usepackage{amssymb}
\usepackage{amsmath}
\usepackage{multicol}

% correct bad hyphenation here
\hyphenation{}

\journal{Journal of Transportation Research Part C: Emerging Technologies}

\begin{document}

\begin{frontmatter}

\title{Indic@: a Cloud-based system for the simulation and testing of the dynamic
pantograph–catenary interaction
\tnoteref{funding}}

\tnotetext[funding]{Research partially supported by project.}

\author[albacete]{Jose A. Mateo\corref{corresponding_author}}
\cortext[corresponding_author]{The corresponding author.}
\ead{JoseAntonio.Mateo@uclm.es}

\author[albacete]{Enrique Arias}
\ead{Enrique.Arias@uclm.es@uclm.es}

\author[albacete]{Tomas Rojo}
\ead{Tomas.Rojo@uclm.es}

\author[albacete]{Fernando Cuartero}
\ead{Fernando.Cuartero@uclm.es}

\address[albacete]{Universidad de Castilla-La Mancha, Albacete, Spain, 02071.}

\begin{abstract}
In this paper, we present a Cloud-based system for simulating and 
testing the dynamic pantograph-catenary interaction.
This high-throughput system called Indic@ represents a disruptive movement in the way 
the catenary-pantograph interaction software is
delivered to the final user (engineering companies). From now, we move 
from desktop applications to Software as a Service (SaaS), 
where users have to forget initial setup costs, paying only 
for the use they consume. Moreover, we have implemented a web platform to
automatize the creation of test scenarios just providing the initial and final
value of the parameter one want to study. In addition to this, these test are run in
a private cloud deployed using the framework OpenNebula and KVM, where the deployment of the
virtual machines and the maintenance of the computing infrastructure 
is completely transparent to the final user. 

\end{abstract}

\begin{keyword}
%% keywords here, in the form: keyword \sep keyword
 Catenary-pantograph interaction \sep Cloud Computing \sep OpenNebula \sep Simulation \sep Software as a Service (SaaS) \sep High-throughput
\end{keyword}

\end{frontmatter}

\section{Introduction}\label{sec:introduction}
In recent years, railway transportation has experienced considerable progress,
where commercial train speed exceeds 300 km/h in many countries.
To achieve this speed, a huge amount of energy (e.g. electricity or gas) is consumed, and,
therefore, it is of great interest to study new methods since energy costs are rising. As
electric train are widely deployed worldwide, we focus here on the interaction between the line and
the train. Especially when considering high speeds, it is
necessary to maintain a constant flow of electricity on the
tensile member for proper operation of railway units. The
pantograph/catenary system based on overhead contact line
(OCL or catenary) is the most widely used for feeding modern trains. 
The catenary is a structure formed by cables that
supply electricity to the pantograph, which is responsible for
obtaining the energy by direct contact through a surface rub,
where the development of a rigorous mechanical calculation
is essential in order to achieve optimal coupling between both
members.

In \cite{}, we introduced the mechanical equations for solving the catenary-interaction problem and two
tools were presented.



The paper is structured as follows 

\section{Background}\label{sec:background}
\section{Cloud Computing}
The NIST also states that Cloud computing is composed by five essential
characteristics. These are: on-demand self-service, broad network access,
resource pooling, rapid elasticity and measured service. Offered as Software
as a Service (SaaS) and/or Platform as a Service (PaaS) and/or Infrastructure
as a Service (IaaS), across four different deployment models: Public, Private,
Community and Hybrid (as a combination of two or the three deployment
models.
More specifically, these five characteristics represent:
\begin{itemize}
\item On-demand self-service: A user can unilaterally provision computing
capabilities or resources (i.e., server time, storage, CPU architecture,
etc.) as needed, without human interaction.
\item Broad network access: The environment must be available over the
network and accessed through standard mechanisms, enabling heterogeneous
platforms access (i.e., mobile phones, laptops, etc.)
\item Resource pooling: Resources must be pooled in order to provide a
multi-tenant model, providing different physical and virtual resources
dynamically according to users demand. Thus, hiding where the resources
are located, and the complexities in order to manage them.
\item Rapid elasticity: The infrastructure resources can be elastically provisioned
and released in order to to promote the scalability.
\item Measured service: Cloud systems need to have metering capabilities
in order to measure the services provided. Mainly driven by economical
exploitation and resource control and optimization. To this end,
resource monitorization and control allows providers to manage the infrastructure
providing transparency to users.
\end{itemize}
These features are provided following three service models:
\begin{itemize}
\item SaaS: This service model provides the capability to use providers applications
on the Cloud infrastructure, hiding the complexities of the
underlying infrastructure and disallowing any access to resources such
as, network, storage, OS, among others.
\item PaaS: Alternatively, this service model provides a more flexible environment,
where the capability of deploying consumer-created or acquired
applications within the Cloud infrastructure is provided. However,
these applications have to be designed and developed with languages,
libraries, services and tools supported by the Cloud provider,
also disallowing the underlying infrastructure management or control,
as in the SaaS model.
\item IaaS: Finally, the most flexible model, allows users to deploy and run
arbitrary software, accessing to processing, storage, network or another 
computing resources available in the Cloud infrastructure, hiding partially
the underlying infrastructure management and control.
\end{itemize}

However, it is possible to find other service models focused on other purposes,
such as Network as a Service or Hardware as a Service, or even specific
applications provided as service models (i.e. Hadoop as a Service (HaaS)).
Finally, four deployment models are considered:
\begin{itemize}
\item Public: The Cloud infrastructure is accessible for open use by general
public.
\item Private: Alternatively, the Cloud infrastructure is only available for
exclusive use by a single organization.
\item Community: This is an intermediate model between the public and
private models, where the Cloud infrastructure is accessible for exclusive
use by a specific community or organizations.
\item Hybrid: The last deployment model, considers the combination of any
of the previous deployment models within the same Cloud infrastructure.
As a result from
\end{itemize}


\section{Software as a Service (SaaS)}
Regarding to SaaS, there are a  number of reasons to adopt this new promising distribution model in industry \cite{}. Next, we present the 
most relevant for the users of our framework:  
    \begin{itemize}		
		\item No additional hardware costs. In our case, users do not assume the cost of the required infrastructure to deploy Indic@. 				Furthermore, they do not have to pay for the renovation of this infrastructure since this is part of the task of the cloud provider. Thus, small companies can use this software without spending a lot of money, helping to speed up these companies at their beginning. 
    \item No initial setup costs. In the past, Indica \cite{} was a Windows desktop application so that for each user we have to install it in each machine, consuming time and human resources. From now on, Indic@ is accesible for all the users through the same interface.  
    \item Pay for what you use; if a piece of software is only needed for a limited period then it is only paid for over that period and subscriptions can usually be halted at any time. Indica was distributed under license so that companies have to pay for every single copy. From now on, they only pay according to the usage of the software, leading to a substantial save of money. 
    \item Usage is scalable; if a user decides they need more storage or additional services. Formerly, this was a drawback of Indica. Thus, users could not run in parallel tens of simulations, having to wait between them. With Indic@, engineers can simulate multiple scenarios at the same time, causing a major time saving.
    \item Updates are automated; whenever there is an update it is available online to existing customers. This one of the main contributions of Indic@. When a new feature (e.g. new calculation method) is implemented, it is instantaneously deployed in the cloud, and, then, it is available to the users. With the other approach, new versions of the tool had to installed again in the computer with the corresponding delay for the users.
    \item Cross device compatibility; SaaS applications can be accessed via any Internet enabled device, which makes it ideal for those who use a number of different devices, such as internet enabled phones and tablets, and those who do not always use the same computer. In our case, this is a key feature. With Indic@, engineers can use the framework in any place and with any device, giving them freedom to choose their setup. As commented, Indica is a Windows application and, therefore, user is tied to this platform.
    \item Accessible from any location; rather than being restricted to installations on individual computers, an application can be accessed from anywhere with an internet enabled device. This a big improvement in Indic@. Normally, engineers have office and field work, and, therefore, it is of great interest to provide mobility solutions in the sense simulations are available worldwide for us and also for our team workers.
    \item Applications can be customized with some software, customization is available meaning it can be altered to suit the needs and branding of a particular customer.
		\end{itemize}
\section{Conclusions and Future Work}\label{sec:conclusions}



%% The Appendices part is started with the command \appendix;
%% appendix sections are then done as normal sections
%% \appendix

\bibliographystyle{elsarticle-num}
\bibliography{bibliografia}

\end{document}
\endinput
